\chapter*{Résumé}
\addcontentsline{toc}{chapter}{Résumé}

Cette thèse vise à comprendre les fondements et les fonctions des calculs probabilistes impliqués dans les processus visuels.  Nous nous appuyons sur une double stratégie, qui implique le développement de modèles dans le cadre du codage prédictif selon le principe de l'énergie libre. 
Ces modèles servent à définir des hypothèses claires sur la fonction neuronale, qui sont testées à l'aide d'enregistrements extracellulaires du cortex visuel primaire. Cette région du cerveau est principalement impliquée dans les calculs sur les unités élémentaires des entrées visuelles naturelles, sous la forme de distributions d'orientations.

Ces distributions probabilistes, par nature, reposent sur le traitement de la moyenne et de la variance d'une entrée visuelle. Alors que les premières ont fait l'objet d'un examen neurobiologique approfondi, les secondes ont été largement négligées. Cette thèse vise à combler cette lacune.

Nous avançons l'idée que la connectivité récurrente intracorticale est parfaitement adaptée au traitement d'une telle variance d'entrées, et nos contributions à cette idée sont multiples. 
(1) Nous fournissons tout d'abord un examen informatique de la structure d'orientation des images naturelles et des stratégies d'encodage neuronal associées. Un modèle empirique clairsemé montre que le code neuronal optimal pour représenter les images naturelles s'appuie sur la variance de l'orientation pour améliorer l'efficacité, la performance et la résilience.  
(2) Cela ouvre la voie à une étude expérimentale des réponses neurales dans le cortex visuel primaire du chat à des stimuli multivariés. Nous découvrons de nouveaux types de neurones fonctionnels, dépendants de la couche corticale, qui peuvent être liés à la connectivité récurrente. 
(3) Nous démontrons que ce traitement de la variance peut être compris comme un graphe dynamique pondéré conditionné par la variance sensorielle, en utilisant des enregistrements du cortex visuel primaire du macaque.
(4) Enfin, nous soutenons l'existence de calculs de variance (prédictifs) en dehors du cortex visuel primaire, par l'intermédiaire du noyau pulvinar du thalamus. Cela ouvre la voie à des études sur les calculs de variance en tant que calculs neuronaux génériques soutenus par la récurrence dans l'ensemble du cortex.


\vspace{0.25cm}
Mots-clés: vision, variance, cerveau Bayesien, encodage prédictif, neurobiologie, neurocomputation, connectivité synaptique récurrente.

\selectlanguage{english}
