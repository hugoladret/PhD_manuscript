\chapter*{Abstract}
\addcontentsline{toc}{chapter}{Abstract}
\selectlanguage{english}
\begin{flushright}
    \textit{''Dear Sir or Madam, will you read my book ?\\
    It took me years to write, will you take a look ?''}\\
    The Beatles, Paperback Writer, 1966
\end{flushright}

This thesis aims to understand the foundations and functions of the probabilistic computations involved in visual processes.  We leverage a two-fold strategy, which involves the development of models within the framework of predictive coding under the free energy principle. 
These models serve to define clear hypotheses of neuronal function, which are tested using extracellular recordings of the primary visual cortex. This brain region is predominantly involved in computations on the elementary units of natural visual inputs, in the form of distributions of oriented edges.

These probabilistic distributions, by nature, rely on processing both the mean and variance of a visual input.
While the former have undergone extensive neurobiological scrutiny, the latter have been largely overlooked. This thesis aims to bridge this knowledge gap.

We put forward the notion that intracortical recurrent connectivity is optimally suited for processing such variance of inputs, and our contributions to this idea are multi-faceted. 
(1) We first provide a computational examination of the orientation structure of natural images and associated neural encoding strategies. An empirical sparse model shows that the optimal neural code for representing natural images relies on orientation variance for increased efficiency, performance, and resilience.  
(2) This paves the way for an experimental investigation of neural responses in the cat's primary visual cortex to multivariate stimuli. We uncover novel, cortical-layer-dependent, functional neuronal types that can be linked to recurrent connectivity. 
(3) We demonstrate that this variance processing can be understood as a dynamical weighted graph conditioned on sensory variance, using macaque primary visual cortex recordings.
(4) Finally, we argue for the existence of (predictive) variance computations outside the primary visual cortex, through the Pulvinar nucleus of the thalamus. This paves the way for studies on variance computations as generic weighting of neural computations, supported by recurrence throughout the entire cortex.

\vspace{0.25cm}
Keywords: vision, variance, Bayesian brain, Predictive Coding, neurobiology, neurocomputation, recurrent cortical connectivity.

\selectlanguage{english}
