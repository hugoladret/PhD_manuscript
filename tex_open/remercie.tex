\chapter*{Acknowledgements}
\addcontentsline{toc}{chapter}{Acknowledgements}
\begin{flushright}
    \textit{''En bande organisée, personne peut nous canaliser''}\\
    13'Organisé, Bande Organisée, 2020
\end{flushright}
The following pages recapitulate half a decade of exciting scientific adventures, a voyage brought to life by the indispensable contributions of many wonderful human beings.

First and foremost, this journey would not have been possible without the support of my two co-supervisors. Laurent, who many years ago embraced the idea of recruiting a molecular biologist to work on a computational project. Embraced it so well, actually, that he did not fire me when I googled "how to import NumPy" on my first day of work. Laurent, your constant presence throughout this journey has been an unwavering beacon, always showing the way for unconventional and bold ideas to pursue together. More than a supervisor, you are a mentor who has instilled in me a passion for the relentless pursuit of scientific discovery, and for that I am forever in your debt. Your generosity with the conference budget allowed to gather the massive "natural images dataset" used in the third chapter of this thesis; but also allowed us to foster many collaborations throughout this thesis, defining this adventure and many new ones to come.\\
Christian was one such collaborator, who then went on to become co-director of this journey. Just as Laurent took a leap of faith by allowing a biologist to embark on a computational project, you have doubled down on that courage by entrusting a newly converted computationalist with the reins of a biological project. Considering the tangible resources (and budget) at play, I'd say this shows a lot of your generous nature, as you welcomed me with open arms and unfaltering support, undeterred by (my many) experimental setbacks, (constant) requests for new computers, (unreasonable) new experimental materials or even (suspiciously frequent) new electrodes. The lessons of scientific independence and resilience learned during our adventure together will last a lifetime. Thank you also for letting me take home a lab's server during the COVID-19 lockdown, as I will now confess that the computational power unclaimed by my code might was also channeled into resource-intensive GPU applications (i.e. \textit{Doom Eternal}) during these two interminable months.
Rounding off my trio of navigators on this scholarly odyssey is Frédéric, who more than deserves the title of unofficial third supervisor. Fredo, I am grateful for your understanding and patience with my naïve questions regarding freshly harvested results, but also for your razor-sharp mind during their analysis, and keen eye when reviewing our papers (yes, I can already hear you say \textit{"fayot !"} as you read these lines). I tried my best to not beat your record of "more paper cited than Fregnac", but I think I failed, although there might be one or two that are in the citation list as jokes.

The larger crew of this ship is composed of numerous individuals whose invaluable contributions have guided this expedition into the brain.
Nelson, thank you for your tutelage in electrophysiology, which was nothing short of phenomenal, and for our \textit{"sensei-professoro"} relationship that has evolved into one of the most cherished friendships of my life. 
Geneviève, you are the source of the magic that makes the Montreal lab alive, and I thank you for investing extra time to unveil the arcane workings of all our experimental tools. Lamyae, my companion in arms as we wrestled a capricious electrophysiology system, best wishes for your own thesis journey, and thank you for having single-handedly recorded $56\%$ (I counted) of the neurons we used in the Nature Communications Biology article. I'm still amazed by your electrode-handling magic ! 
To the eminent scientists who played advisory roles in this thesis – David, Karim, Paul, and Steve – thank you for sharing the burden of administrative constraints with grace and fortitude, turning these hurdles into stepping stones on my path. Many thanks to the members of the jury, who have agreed to read a manuscript that is longer than it has any right to be. I could not think of anyone better suited than those who have gracefully agreed to sit on this defense, and count myself extremely lucky to have such eminent future "peers". \\
There is a rather lenghty list of people who contributed significantly to this thesis. Guillaume, thank you for a pinpoint accurate critique of a presentation I delivered years ago, which served as a wake-up call to catalyze a journey towards higher scientific standards. 
Guilhem, thank you for our (brief) collaboration in modeling the parietal cortex during the early stages of my PhD.
Jonathan, thank you for having paved the way for all things related to the involvement of Motion Clouds in the world of neurophysiology.
%Jason, many thanks for mixing camaraderie and engineering rigor together, and for keeping private our "meme" chat, unless we walk the plank from this here scientific vessel. 
Antoine, your kindness, gourmet barbecues, and expert knowledge of logistic regression have made you a cherished colleague and friend.
Jean-Nicolas, thank you for your refreshing sense of humor, our shared appreciation for what others might call disgusting stout beers, and for invigorating debates on deep learning.
Patrice and Marion, for the many "apéros" and incredible luck of having you as friends, but also for endless exchanges of TP after one "apéro" too many. Louison, for walking through the callanques at hellish speeds, I owe you Figure 1 of Chapter 4's article ! On the mention of hiking, special thanks to Shashank, for having had the kindness of driving all the way to the end of Telluride's valley to pick me up, saving my toes and earning the right to name my firstborn (I hope you won't go too crazy).
Justine, for transitioning from best friend to also being a colleague - hopefully, you'll have forgiven me for any monkey business.
Marjory and Matthew, for a friendship that has endured the cold winter of Canada and the sunny shores of Nice... but eventually decided to move back to the cold winter part.
Samuel Chéri, pour des soirées de rigolades inoubliables dans le froid du Québec (t'as vu, j'ai écris en français exprès pour toi).\\
My thanks are also to Salvatore, Mohit, Sandrine, Kevin and Ivo for the camaraderie of being a NeOpTer. To Anna for insider knowledge (all in good fun) about Neuroschool, and for her constant cheerfulness. To the upcoming generation of INT scientists who've set incredibly high standards : Cléo, Uriel, Miles, Marie. To the SYRT : Lucie, Montaine, Alexis and Lucio. To Taarabte, whose resourcefulness was nothing short of a lifeline in saving my salary for the last year. To Hélène, Joelle, Céline, your collective kindness has been a shield against my administrative faux pas, and I am eternally grateful.
To scientists outside my labs : Cyril Kurz, Ingrid Bureau, Melissa Mardelle, Emna Marouna, Jérémy Camon; Francesca Sargolini, Frank Vidal, Christian Gestreau – your influence has been crucial in setting me onto this path.
To those who acted as lighthouses on this scientific voyage: the remarkable 2022 Schloss Rauischholzhausen crew (hello Crème de la Crème), the brilliant minds of the 2023 Telluride Neuromorphic Workshop (here's looking at you, Condo 301, the residence of champions), and the many other talented individuals I've had the good fortune of meeting at various scientific events.
In that spirit, a final scientific salute goes to Tony Movshon, Dario Ringach, Peter Roelfsema, and the extraordinary Paolo Papale, for fruitful scientific exchanges that still carry impactful consequences on this day.

Finally, there remains the matter to acknowledge those who were not on the ship, but have offered unwavering support from the shore.
To my dad, who merits the first position in this section of the acknowledgements, because the inspired 'Paperback Writer' quote was stolen from him, and also for pushing me to try this "science thing". Equally, to my mom, who is second in sequence but not in importance, who never faltered in her support to me and kept my cheese and happiness supply steady. I could have never done this without you !
To my girlfriend, who stood by me during the final push towards the finish line and tried to talk me out of my new cowboy hat-wearing habit since my return from Telluride.
And then, of course, to my friends, ranked according to how much I appreciate them: Alexandre, Anthony, Arman, Axel, Cédric, Cem, Chloé, Dan, Davidou, Elysa, Florian, JB, Jérémie, Jessica, Louison, Manu, Margot, Martine, Michou, Nicolas, Perrine, Raphael, Rémi, Rich, Thomas, Tony, Victoria, Yannis and Zoé (I am of course kidding, that was in alphabetical order).

On a reflective note, and an important one, I want to express my gratitude to Umit Keysan for the memorable moments we shared. We playfully challenged each other to see who would earn their PhD first. While fate took an unexpected turn, the journey was enriched by your camaraderie. Your spirit continues to inspire me, and I miss you, my brother.

Last, but not least, thank you for reading this manuscript. Whether you are now reviewing a fresh copy before my defense, or are years later being forced to sift through every page to continue my project in one of the labs I was part of, thank you for breathing life once more into this work. I hope you will enjoy the reading as much as I enjoyed the \textit{science-ing}.